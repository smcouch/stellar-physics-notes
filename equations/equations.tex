% !TEX root = ../stellar-notes.tex

\section{The conservation laws}

After our rapid overview, we now gather the tools needed to tackle stellar evolution.  The first is to get the macroscopic equations for stellar structure. We will start from the equations expressing conservation of mass\footnote{In a relativistic system, we would instead start from conservation of baryon number, since mass is not invariant.}, momentum, and energy. We already derived the continuity (conservation of mass) equation,
\begin{equation}\label{e.mass-1}
\partial_{t}\rho + \divr(\rho\vu) = 0,
\end{equation}
and the Euler equation,
\begin{equation}\label{e.momentum-1}
\partial_{t}\vu + \vu\cdot\grad\vu = -\grad \Phi - \frac{1}{\rho}\grad P.
\end{equation}
Note that if we multiply eq.~(\ref{e.momentum-1}) by $\rho$, we can rewrite it, using eq.~(\ref{e.mass-1}), as
\begin{equation}\label{e.momentum-2}
	\partial_{t}(\rho\vu) + \divr[\vu(\rho\vu)] = -\rho\grad\Phi -\grad P.
\end{equation}
The left-hand side is interpreted as expressing the conservation of momentum ($\rho\vu$) in the absence of forces, analogous to eq.~(\ref{e.mass-1}) for the conservation of mass ($\rho$).

Note the general form of a conservation equation:
\begin{eqnarray*}
\lefteqn{\partial_{t}(\textrm{conserved quantity})}\\
 & & \mbox{} + \divr(\textrm{flux of conserved quantity}) =
 (\textrm{sources}) - (\textrm{sinks}).
\end{eqnarray*}
Because the momentum density $\rho\bvec{u}$ is a vector, its flux is a tensor: $[\bvec{u}(\rho\bvec{u})]_{ij} \equiv \rho u_{i}u_{j}$.\marginnote{By $\rho u_{i}u_{j}$, we mean the momentum along direction $i$ being transported along direction $j$.}

\newthought{The next equation is that of energy conservation.} Here we must consider both the internal energy per unit volume $E/V = \rho \varepsilon$ and the kinetic energy per unit volume $\rho u^{2}/2$.  In this section $\varepsilon$ represents the internal energy per unit mass of the fluid. In a fixed volume of the fluid the total energy is thus
\[ \int_{V} \left(\rho \frac{1}{2}u^{2} + \rho \varepsilon \right)\,\dif V. \]
The flux of energy into this volume will clearly include
\[ -\int_{\partial V}\left(\frac{1}{2}\rho u^{2} + \rho \varepsilon\right) \vu\vdot\dif\bvec{S}. \]
But wait, there's more!  In addition, we have a conductive heat flux $\bvec{F}$; the total heat conducted through the surface $\partial V$ is
\[-\int_{\partial V}\bvec{F}\vdot\dif\bvec{S}.\]
Moreover, the pressure acting on fluid flowing into our volume does work on the gas at a rate
\[-\int_{\partial V}P\vu\vdot\dif\bvec{S}.\]
As a result, the net change of energy in our volume is
\begin{eqnarray}
\lefteqn{\partial_{t}\int_{V}\left(\frac{1}{2}\rho u^{2} + \rho \varepsilon\right)\nsp\dif V = }\nonumber \\
 && -\int_{\partial V} \!\dif\bvec{S}\vdot\left[\vu\left(\frac{1}{2}\rho u^{2} + \rho \varepsilon + P\right) + \bvec{F}\right] \nonumber\\
 &&+ \int_{V} \left( \rho \vu\vdot\bvec{g} + \rho q\right)\dif V.
\label{e.energy-1}
\end{eqnarray}
On the right-hand side we've added in the work done by gravity and the heating evolved by nuclear reactions (this could also involve sinks, such as neutrinos with a long mean free path).
Expressed in differential form, equation~(\ref{e.energy-1}) is
\begin{equation}\label{e.energy-2}
 \partial_{t}\left(\frac{1}{2}\rho u^{2} + \rho \varepsilon\right)
 	+ \divr\left[\rho\vu\left(\frac{1}{2} u^{2} + \varepsilon + \frac{P}{\rho}\right)\right]
	+ \divr\bvec{F} = \rho q + \rho \vu\vdot\bvec{g}.
\end{equation}
You are possibly wondering why I didn't put gravity, which can be expressed as a potential, on the left hand side of this equation.  The reason is that the gravitational stresses cannot be expressed in a  \emph{locally} conservative form; it is only when integrating over all space that the conservation law appears.

\newthought{Equations~(\ref{e.mass-1}), (\ref{e.momentum-2}), and (\ref{e.energy-2}) are supplemented by an equation of state,} which allows one to get from the pressure $P$, the temperature $T$, and the mass fractions $X_{i}$ of the species present, the remaining thermodynamical quantities, such as mass density $\rho$ and specific energy $\varepsilon$. In addition, Poisson's equation
\begin{equation}\label{e.poisson}
\nabla^{2}\Phi = 4\pi G\rho,
\end{equation}
specifies the gravitational acceleration $\bvec{g} = -\grad\Phi$. We then need one more equation to specify the heat flux $F$. We argued in \S\ref{s.energy-transport-estimate} that the typical length over which a photon travels before scattering is very small compared to the lengthscale over which the macroscopic properties of the star vary.  In this case, we expect the flux to obey a conduction equation of the form
\begin{equation}\label{e.conduction-simple}
\bvec{F} = -K\grad T.
\end{equation}
This assumption is clearly questionable near the stellar surface, and we have left unspecified the form of $K$.  Such an equation does, however, close the system of equations; all of the physics is then contained in the equation of state $P(\rho,T,\{X_{i}\})$, the rate of heating from nuclear reactions $q(\rho, T, \{X_{i}\})$, and the thermal conductivity $K(\rho,T,\{X_{i}\})$. Here $\{X_{i}\}$ are the mass fractions of the isotopes composing the solar plasma. We will also need a system of equations to describe how the $X_{i}$ change as a result of nuclear reactions and diffusion.


\section[Thermodynamics of a Mixture]{Thermodynamics of a mixture: A digression}\label{s.composition}
\subsection{Specifying the composition}
In this section we'll look at how one describes the composition for a multi-component plasma.  To make things concrete, let's imagine a box containing a mixture of nuclei, of many different isotopes, and electrons.  (To keep things simple, we'll assume complete ionization.)  Each isotope species $i$ has $N_{i}$ nuclei present, and is characterized by charge number $Z_{i}$ and nucleon number $A_{i}$.  Charge neutrality then specifies the number of electrons,
\begin{equation}\label{e.number-e}
N_{\mathrm{e}} = \sum_{i} Z_{i} N_{i}.
\end{equation}
The total mass of the box is
\begin{equation}\label{e.total-mass}
M = \me N_{\mathrm{e}} + \sum_{i} m_{i}N_{i},
\end{equation}
where $\me$ and $m_{i}$ are respectively the mass of an electron and a nucleus of species $i$.  Now what is $m_{i}$? Breaking a nucleus $i$ into $Z_{i}$ protons and $A_{i}-Z_{i}$ neutrons takes a certain amount of energy, the \emph{binding energy} $B_{i}$.  We can therefore write $m_{i} = Z_{i}\mpr + (A_{i}-Z_{i})\mn - B_{i}/c^{2}$, where $\mpr$ and $\mn$ are respectively the proton and neutron rest masses.

Inserting our expression for $m_{i}$ into equation~(\ref{e.total-mass}), dividing by the volume of the box $V$, and rearranging terms gives us the mass density,
\begin{equation}\label{e.rho}
\rho = \frac{M}{V} = \sum_{i} n_{i}\left[ \left(A_{i}-Z_{i}\right) \mn + Z_{i}\left(\mpr + \me\right) - B_{i}/c^{2}\right].
\end{equation}
Here $n_{i}$ is the number density of isotope species $i$, and we have used equation~(\ref{e.number-e}) to eliminate $N_{\mathrm{e}}$.  The numbers $n_{i}$ are, of course, fantastically\sidenote{astronomically?} large, so we scale the numbers by \emph{Avogadro's constant},
\begin{equation}\label{e.avogadro-def}
\NA = 6.0221367\ee{23}\nsp\unitstyle{mol}^{-1}.
\end{equation}
\marginnote{Recall that a \textbf{mole} is an amount of something: in $1\nsp\unitstyle{mol}$ there are $\NA$ items.}
If we multiply and divide the right-hand side of equation~(\ref{e.rho}) by $\NA$, we then have
\begin{equation}\label{e.molar-1}
\rho = \sum_{i} \left(\frac{n_{i}}{\NA}\right) \mathcal{A}_{i},
\end{equation}
where
\begin{equation}\label{e.gm-mol}
\mathcal{A}_{i} = \left[ \left(A_{i}-Z_{i}\right) \mn + Z_{i}\left(\mpr + \me\right) - B_{i}/c^{2}\right]\times\NA
\end{equation}
is the \emph{gram-molecular weight} of species $i$ with dimensions $[\mathcal{A}]\sim[\gram\cdot\mol^{-1}]$. Strictly speaking, the gram-molecular weight actually refers to the mass of a mole of the isotope in \emph{atomic} form; the right-hand side of eq.~(\ref{e.gm-mol}) is the gram-molecular weight neglecting the electronic binding energy.

Now you may wonder where the numerical value of $\NA$ came from.  It is not pulled out of thin air, but rather is defined so that \val{1}{\mol} of \carbon\ has a mass of exactly \val{12}{\gram}.  In other words, for \carbon\, $\mathcal{A} \equiv A\nsp\gram\usp\mol^{-1}$.  In fact for all nuclei, $\mathcal{A} \approx A \usp\gram\nsp\mol^{-1}$ to better than about 1\%, as demonstrated in Table~\ref{t.gm-mol}.
Because in CGS $\mathcal{A}\approx A$, it is customary to write $\mathcal{A} = A\times (1\usp\gram\nsp\mol^{-1})$, so that equation~(\ref{e.molar-1}) is
\begin{equation}\label{e.molar-2}
\rho = \sum_{i} \left(\frac{n_{i}}{\NA}\times 1\frac{\gram}{\mol}\right) A_{i}.
\end{equation}
This only works if our unit of mass is the gram: in SI units \carbon\ has a mass of $0.012\nsp\kilo\gram\usp\unitstyle{mol}^{-1}$.
Equation~(\ref{e.molar-2}) would be exact if $A$ were a real number, but the custom is to just keep it as the nucleon number, which introduces an error of order one percent. Astronomers typically then \emph{redefine} $\NA$ to mean $\NA(\textrm{astronomy})\equiv \NA / (1\nsp\gram\usp\mol^{-1}) = 6.0221367\ee{23}\nsp\gram^{-1}$. Alternatively, one can use the atomic mass unit (symbol $\amu$) defined as $1/12$ the mass of an atom of \carbon, so that $1\nsp\amu =  (1\nsp\gram\usp\mol^{-1})/\NA = 1.66054\ee{-24}\nsp\gram$. This puts equation~(\ref{e.molar-2}) into the more obvious form $\rho = \sum n_{i}\times A_{i}\mb$, with $\mb$ having a mass of $1\nsp\amu$.

\begin{table}[htbp]\caption{\label{t.gm-mol}Selected gram-molecular weights.}
\begin{center}
\begin{tabular}{r|ccc}
\hline
nuclide & $A$ & $\mathcal{A}$ & $(|\mathcal{A}-A|/A) \times 100$\\
\hline\hline
\neutron & 1 & 1.00865 & 0.865\\
\hydrogen & 1 & 1.00783 & 0.783\\
\helium & 4 & 4.00260 & 0.065\\
\oxygen & 16 & 15.99491 & 0.032\\
\silicon & 28 & 27.97693 & 0.082\\
\iron & 56 & 55.93494 & 0.116\\
\hline
\end{tabular}\end{center}
\end{table}

With the redefinition of \NA, equation~(\ref{e.molar-2}) can be rewritten as
\begin{equation}\label{e.}
1 = \sum_{i}\left(\frac{n_{i}}{\NA\rho}\right)A_{i} \equiv \sum_{i}Y_{i}A_{i}
\end{equation}
where $Y_{i} \equiv n_{i}/(\rho\NA)$ is the \emph{molar fraction}. It is customary to call $Y_{i}A_{i}$ the \emph{mass fraction} $X_{i}$, with $\sum X_{i} = 1$. We can then define the mean atomic mass number,
\begin{equation}\label{e.mean-A}
\bar{A} = \frac{\sum A_{i}Y_{i}}{\sum Y_{i}} = \frac{1}{\sum Y_{i}},
\end{equation}
and mean charge number
\begin{equation}\label{e.mean-Z}
\bar{Z} = \frac{\sum Z_{i}Y_{i}}{\sum Y_{i}} = \bar{A} \sum Z_{i}Y_{i}.
\end{equation}
The molar fraction of electrons is
\begin{equation}\label{e.Ye}
Y_{e} = \sum Z_{i} \frac{n_{i}}{\rho\NA} = \sum Z_{i}Y_{i} = \frac{\bar{Z}}{\bar{A}}.
\end{equation}
In stellar structure work, it is common to use the \emph{mean molecular weight}, defined so that the total number of particles, including electrons, per unit volume is
\begin{equation}\label{e.mean-molecular-weight}
\sum_{i} n_{i} + n_{e} \equiv \frac{\rho\NA}{\mu}.
\end{equation}
Yes, this is still the redefined $\NA$: $\mu$ is dimensionless. From the definition,
\begin{equation}
\mu = \left(\sum_{i}Y_{i} + Y_{e}\right)^{-1} = \left[ \sum_{i}\left(Z_{i}+1\right)Y_{i} \right]^{-1};\label{e.meanmol}
\end{equation}
sometimes astronomers also define the mean ion molecular weight, $\mu_{I} = (\sum Y_{i})^{-1}$, and the mean electron weight, $\mu_{e} = Y_{e}^{-1}$.

\begin{exercisebox}[Mean molecular weight of a hydrogen-helium plasma]
Consider a gas of \hydrogen\ and \helium\ with molar hydrogen fraction $Y_{\mathrm{H}}$.  Derive expressions for the molar fraction of \helium, $Y_{\mathrm{He}}$, $\bar{A}$, $\bar{Z}$, and $\mu$. What are the numerical value of these quantities for $Y_{\mathrm{H}} = 0.7$, i.~e., solar?
\end{exercisebox}

\begin{exercisebox}[Sound speed and kinetic energy of an ideal gas]
Assume that we can describe this plasma as an ideal gas.  What is the sound speed and the average kinetic energy of a particle, for a given mass density and temperature?
\end{exercisebox}


%In Ch.~\ref{ch.introduction} we listed the abundances of several elements in the sun: for $A_{\mathrm{el}} \equiv \log_{10}[N_{\mathrm{el}}/N_{\mathrm{H}}]+12$, $A_{\mathrm{H}} = 12.00$, $A_{\mathrm{He}} = 10.93$, $A_{\mathrm{N}}=7.78$, $A_{\mathrm{O}}=8.66$, $A_{\mathrm{C}} = 8.39$.  Assume that the isotopes for each element are, respectively, \hydrogen, \helium, \nitrogen, \oxygen, and \carbon; from this, derive the corresponding mass fractions for each isotope.

\section{The Equation of State of an Ideal Gas} \label{s.idealgas}
Equations~(\ref{e.mass-1}), (\ref{e.momentum-2}), and (\ref{e.energy-2} must be closed by a suitable {\it equation of state} that relates the thermodynamic quantities to each other.
A simple equation of state is that of an ideal gas.
An {\it ideal gas} is assumed to be made of vast numbers of point-like particles that interact only via elastic collisions.
This is generally a pretty good approximation for a monatomic, single-species gas but also works fairly well for molecular real gasses such as the terrestrial atmosphere, within certain limits.
Typically, a gas behaves more ideal at higher temperatures at which the kinetic motion of the particles dominate any intermolecular forces (i.e., van der Waals) that the gas may experience.
Ideal gasses obey the {\it ideal gas law}.
A {\it perfect gas} is a special case of an ideal gas that has constant specific heat capacities at constant volume and at constant pressure (see Section \ref{s.thermodynamics}).
In general, an ideal gas can have specific heat capacities that temperature- and/or density-dependent.

Consider a closed volume containing a large number of point-like collisional particles.
Elastic collisions of these particles with the walls of the volume will result in a force exerted on the walls.
In the absence of any external forces (i.e., gravity), the force exerted will be the same for all the walls of the volume.
The force per wall area is the {\it pressure} P on the walls.

The ideal gas law was first determined empirically by chemists in the 19th century and relates the pressure of a gas to its volume and temperature:
\begin{equation}
	PV = N_m R T, \label{e.ideal1}
\end{equation}
where $V$ is the volume of the gas, $T$ its temperature, $N_m$ the number of {\it moles} of gas, and $R$ is the universal gas constant.
It turns out, experimentally, the number of particles in a mole of gas is a universal constant: Avogadro's number \NA.
The universal gas constant is the product of Avogadro's number and the Boltzmann constant \kB: $R=\NA \kB$.
Thus, we can write the ideal gas law in a form more familiar to physicists,
\begin{equation}
	P = n\kB T,\label{e.ideal2}
\end{equation}
where $n = N_M\NA/V$ is the number density of particles in the gas.

If the particles can be described by a mean atomic mass $\bar{A}$ (Eq. \ref{e.mean-A}) then we can express the mass density in terms of the number density as $\rho = n A \mb$, where \mb is the atomic mass unit. Then another form of the ideal gas law is
\begin{equation}
	P = \frac{\rho}{\bar{A} \mb} \kB T.\label{e.ideal3}
\end{equation}

For a mixture of different particle types, including free electrons as we would have in a stellar plasma, we can express the ideal gas law in yet another equivalent form using the mean molecular weight,
\begin{equation}
	P = \frac{\NA \rho}{\mu} \kB T = \frac{\rho}{\mu \mb} \kB T,\label{e.ideal4}
\end{equation}
where we recognize that $\NA \approx \mb^{-1}$.
Using the definition of the mean molecular mass in terms of the molar fractions (Eq. \ref{e.meanmol}), we see that the total pressure is the sum of partial pressures,
\begin{equation}
	P = \left(\sum_{i}Y_{i} + Y_{e}\right) \NA \kB \rho T.\label{e.ideal5}
\end{equation}

Equations (\ref{e.ideal1}-\ref{e.ideal5}) are all equivalent for a gas of identical particles, with Eqs. (\ref{e.ideal4}) and (\ref{e.ideal5}) being more general for a mixture that includes free electrons as well as ions.

\section{Thermodynamics of Stellar Plasmas} \label{s.thermodynamics}

\subsection{The First Law}
Consider a volume of gas that can exchange energy with its surrounding either by the transfer of heat or by doing {\it work}.
Now assume the gas undergoes a truly {\it reversible} process, in which the state of the gas changes slowly enough that it is always in equilibrium.
I.e., throughout the entire process the gas moves through a series of infinitesimally different equilibrium states.
In such a process, the change in the gas internal energy will be
\begin{equation}
	\dif E = \dif Q - \dif W,
\end{equation}
where $dQ$ is the heat gained or lost and $dW$ is the work done {\it by} the gas (positive sign) or {\it on} the gas by its surroundings (negative sign).

First, let's consider how to deal with the work term.
If the gas is ``contained'' within some volume $V$, the gas exerts a differential force on the ``wall'' of this volume,
\[
	\delta \bvec{F} = P \delta A \hat{n},
\]
where $\delta A$ is some differential area element of the volume walls.
Now consider that the force exerted by the gas causes an infinitesimal displacement $\dif \bvec{x}$ in the volume walls doing some differential {\it work} (i.e., force through a distance),
\[
	\delta (\dif W) = \delta \bvec{F}\cdot\dif\bvec{x} = P (\hat{n}\cdot\dif\bvec{x})\delta A = P \delta V.
\]
Now, integrating over all area of the volume walls gives us $\dif W = P\dif V$, which you will recall elementary dynamics.
Then we have the First Law of Thermodynamics,
\begin{equation}
	\dif E = \dif Q - P\dif V. \label{e.first-law-extensive}
\end{equation}
\marginnote{First Law of Thermodynamics}

In Eq. (\ref{e.first-law-extensive}) the energy $E$ and heat $Q$ are {\it extensive} quantities that scale with the number of particles $N$ in our sample. In a fluid, however, these quantities are all functions of position. By $E(r)$, we mean that we can define a small portion of the star about the coordinate $r$ that is large enough particles to ensure that quantities such as pressure and temperature are well-defined, but small enough that we can treat $E(r)$ as a continuous function of position when integrating over the whole star.
\marginnote{Extensive vs. intensive quantities}

Using extensive quantities in fluid mechanics is cumbersome, so we instead use quantities like the energy per unit mass $\varepsilon  = E/(\rho V)$ or the heat per unit mass $q = Q/(\rho V)$. Since a fixed mass of fluid $M$ occupies a volume $V = M/\rho$, we can divide the first law, eq.~(\ref{e.first-law-extensive}), by $M$ to obtain
\begin{equation}\label{e.first-law-intensive}
\dif \varepsilon = \dif q - P\dif\left(\frac{1}{\rho}\right) = \dif q + \frac{P}{\rho^{2}}\dif \rho.
\end{equation}
The other extensive variables can be re-defined into mass-specific forms in a similar fashion.

The definition of an {\it adiabatic} process is one in which there is no change in the heat content of the system, $\dif Q \equiv 0$.
\marginnote{Adiabatic process}
Consider the Joule-Kelvin experiment in which a gas is allowed to expand adiabatically ($\dif Q=0$) into a larger volume initially a vacuum (i.e., $V_1 < V_2$).
Since the gas experiences no resistance while expanding into the new volume it does not do any work while expanding, $\dif W = P\dif V = 0$.
Therefore, by the First Law (Eq. \ref{e.first-law-extensive}), we have $\dif E = 0$ implying that the temperature remains constant during the expansion, $T_2 = T_1$.
Thus, the internal energy of the gas is a function only of temperature, $E(T)$.
In general, we would expect that the internal energy is also a function of density (or volume), $E(V,T)$.
This somewhat surprising result only holds for perfect gasses.

\subsection{Heat Capacity and Thermodynamic Derivatives}

Now consider a non-adiabtic process in which some heat, $\dif q$, is added to the system resulting in a temperature change $\dif T$.
The {\it specific heat capacity}, or simply specific heat, is then defined as
\[
	C \equiv \frac{\dif q}{\dif T},
\]
where, again, the heat per mass is $q$.
Specific heats are usually specified under conditions in which a thermodynamic quantity is assumed constant.
For instance, the specific heat at constant pressure is $C_P$ and the specific heat at constant volume is $C_V$.

A key feature of thermodynamics is that any thermodynamic quantity can be expressed as a function of any other {\it two} thermodynamic quantity.
Consider, for instance, the ideal gas law (e.g., Eq. \ref{e.ideal3}), which can obviously be arranged as we like to give $P(\rho, T)$, or $T(\rho, P)$, or $\rho(P,T)$.
Thus, it is useful to define derivatives of thermodynamic quantities with respect to others while keeping a third quantity fixed.
For variables $x, y, z$ we would write this using the notation,
\[
	\left(\frac{\partial z}{\partial x}\right)_y,
\]
where the subscript $y$ is the variable held fixed while calculating the derivative of $z$ with respect to $x$.

Now assume that our variables $(x,y,z)$ can be connected by some function $\mathcal{F}(x,y,z)=0$ (such as you might get by moving the RHS of the ideal gas law to the LHS).
Then, we can derive several useful relations (see Appendix \ref{s.thermo-derivatives}):
\begin{eqnarray}
	\left(\frac{\partial x}{\partial y}\right)_z &=& \frac{1}{(\partial y/\partial x)_z}, \\
	\left(\frac{\partial x}{\partial y}\right)_z \left(\frac{\partial y}{\partial z}\right)_x \left(\frac{\partial z}{\partial x}\right)_y &=& -1.
\end{eqnarray}
Now if some other quantity, $w$, is a function of any two of $(x,y,z)$, then
\begin{equation}
	\left(\frac{\partial x}{\partial y}\right)_w \left(\frac{\partial y}{\partial z}\right)_w = \left(\frac{\partial x}{\partial z}\right)_w.
\end{equation}

Consider as independent variables $(\rho, T, \varepsilon)$. Expanding the total differential of $\varepsilon$ in terms of its partial derivatives w.r.t. $\rho$ and $T$, we have
\[
	\dif \varepsilon = \left(\frac{\partial \varepsilon}{\partial T}\right)_\rho \dif T + \left(\frac{\partial \varepsilon}{\partial \rho}\right)_T \dif \rho.
\]
Using the First Law (Eq. \ref{e.first-law-intensive}) we then find
\begin{equation}
	\dif q = \left(\frac{\partial \varepsilon}{\partial T}\right)_\rho \dif T + \left[\left(\frac{\partial \varepsilon}{\partial rho}\right)_T - \left(\frac{P}{\rho^2}\right)\right] \dif \rho. \label{e.firstLaw2}
\end{equation}

For a process occurring at constant volume, the density is also constant\sidenote{Unless we are creating and destroying mass!}, $\dif \rho=0$.
The change in heat is given by $\dif q = C_V \dif T$.
Then by inspection, we find that the specific heat at constant volume is
\begin{equation}
	C_V = \left(\frac{\partial \varepsilon}{\partial T}\right)_\rho. \label{e.specHeatV}
\end{equation}
This is result is {\it completely general} for any thermodynamic expression relating $(\rho, T, e)$.

Similarly, at constant pressure the heat change is $\dif q = C_P \dif T$, and the specific heat is
\begin{equation}
	C_P = C_V + \left[\left(\frac{\partial \varepsilon}{\partial \rho}\right)_T - \left(\frac{P}{\rho^2}\right) \right] \left(\frac{\partial \rho}{\partial T}\right)_P. \label{e.specHeatP}
\end{equation}

For an adiabatic process, we have $\dif q = 0$ and can rearrange Eq. (\ref{e.firstLaw2}) to find
\begin{eqnarray}
		\left(\frac{\partial \varepsilon}{\partial T}\right)_\rho \dif T &+& \left[\left(\frac{\partial \varepsilon}{\partial \rho}\right)_T - \left(\frac{P}{\rho^2}\right)\right] \dif \rho = 0 \nonumber \\
		 \left(\frac{\partial \varepsilon}{\partial T}\right)_\rho \dif T &=& \left[\left(\frac{P}{\rho^2}\right) -\left(\frac{\partial \varepsilon}{\partial \rho}\right)_T \right] \dif \rho  \nonumber\\
		 \left(\frac{\partial \varepsilon}{\partial T}\right)_\rho \left(\frac{\partial T}{\partial \rho}\right)_s  &=& \left[\left(\frac{P}{\rho^2}\right) -\left(\frac{\partial \varepsilon}{\partial \rho}\right)_T \right] \nonumber\\
		 C_V \left(\frac{\partial T}{\partial \rho}\right)_s  &=& \left[\left(\frac{P}{\rho^2}\right) -\left(\frac{\partial \varepsilon}{\partial \rho}\right)_T \right]
\end{eqnarray}
where we have used the definition of the specific heat at constant volume (Eq. \ref{e.specHeatV}) and transformed the full derivative $(\dif T/\dif \rho)$ into a partial derivative at constant entropy, $s$.

The specific heat at constant pressure, Eq. (\ref{e.specHeatP}), is not nearly as simple and neat an expression as that for constant volume, Eq. (\ref{e.specHeatV}).
We can gain some physical insight into the specific heat at constant pressure by defining the specific {\it enthalpy}:
\begin{equation}
	h = \varepsilon + PV = \varepsilon + \frac{P}{\rho}.
\end{equation}
The change in enthalpy in some thermodynamic process is then
\[
	\dif h = \dif \varepsilon + P \dif(1/\rho) + \dif P / \rho
\]
Substituting into the First Law, Eq. (\ref{e.first-law-intensive}), we find
\[
	\dif q = \dif h - \frac{\dif P}{\rho}.
\]
Therefor, at constant pressure ($\dif P=0$) we recognize that we can write the specific heat by replace $q$ with $h$:
\begin{equation}
	C_P = \left(\frac{\partial h}{\partial T}\right)_P,
\end{equation}
a much simpler expression than Eq. (\ref{e.specHeatP}).
This implies that for isobaric processes, the enthalpy takes the role of internal energy for isovolumetric processes. \marginnote{Neat!}

\subsection{The Second Law}

Empirically, it is found that not all energy-conserving cyclic processes are possible.
That is, energy in such processes is transferred into, e.g., its surroundings and is rendered unavailable to do work or heat.
In a sense, the original energy of they system is {\it degraded} and, even if total energy is conserved, it is not possible to return the system to its original state.\marginnote{The process is irreversible.}

To understand this behavior of gasses, we must first introduce a new thermodynamic state function, {\it entropy} $S$.
If in a reversible process a system exchanges some heat $\dif Q$ with a reservoir at temperate $T$, the than change in entropy is
\begin{equation}
	\dif S \equiv \frac{\dif Q}{T}. \label{e.difS}
\end{equation}
Using this definition of entropy, we can re-write the First Law (Eqs. \ref{e.first-law-extensive} and \ref{e.first-law-intensive}) as
\begin{eqnarray}
	T \dif S = \dif E + P \dif V \\
	T \dif s = \dif \varepsilon + \frac{P}{\rho^2}\dif\rho,
\end{eqnarray}
where the second version is the intensive form as we have entropy the {\it specific entropy} $s$.

For any cyclic process, the integral of Eq. (\ref{e.difS}) over an entire cycle must to be less than equal to zero. That is
\begin{equation}
	\oint \frac{\dif Q}{T} \leq 0. \label{e.sec-law-1}
\end{equation}
This is an elementary state of the Second Law of Thermodynamics.

To elucidate the meaning of this further, consider a reversible process in which heat $\dif Q_1 (T)$ is delivered to the system.
In order to reverse the process and return the system to its original state, heat $\dif Q_2 (T) \equiv -\dif Q_1 (T)$ must be given to the system. Equation (\ref{e.difS}) requires that any cycle of the system satisfies both
\[
	\oint \frac{\dif Q_1 (T)}{T} \leq 0
\]
and
\[
	\oint \frac{\dif Q_2 (T)}{T} = -\oint \frac{\dif Q_1(T)}{T} \leq 0.
\]
In order for both conditions to be simultaneously true, the heat transfer must be zero so that for a reversible process
\begin{equation}
	\oint_\mathrm{rev} \frac{\dif Q}{T} = 0. \label{e.rev-dQ}
\end{equation}
This then implies that $\dif S = 0$ by Eq. (\ref{e.difS}).
I.e., in a reversible process, the entropy remains constant.

It will turn out that Eq. (\ref{e.difS}) is an {\it exact} differential, i.e. $\int dS$ is path-independent.\marginnote{An exact differential is one that can be express $\dif f = (\partial f / \partial x)_y \dif x + (\partial x / \partial y)_x \dif y$.}
In other words, the entropy difference between two states is independent of the details of the reversible process connecting them.
If $A$ and $B$ are two states connected by a reversible cycle made up of path 1 ($A$ to $B$) and path 2 ($B$ to $A$) then Eq. (\ref{e.rev-dQ}) implies
\begin{equation}
	\oint \frac{\dif Q}{T} = \left(\int_A^B\frac{\dif Q}{T}\right)_1 + \left(\int_B^A\frac{\dif Q}{T}\right)_2 = 0.
\end{equation}
Now that we've specified some limits we can write these integrals in terms of the entropies of the states via Eq. (\ref{e.difS})
\[
	\left(\int_A^B\frac{\dif Q}{T}\right)_1 = S(B) - S(A),
\]
which gives us
\begin{equation}
	S(B) - S(A) = -\left(\int_B^A\frac{\dif Q}{T}\right)_2 = \left(\int_A^B\frac{\dif Q}{T}\right)_2,
\end{equation}
which proves that the entropy difference of the states doesn't depend on the details of the processes.

Now let's consider that an {\it irreversible} process $I$ connects $A$ to $B$ then reversible process $R$ returns $B$ to $A$.
Now via Eq. (\ref{e.sec-law-1})
\begin{equation}
	\int_I \frac{\dif Q}{T} + \int_R\frac{\dif Q}{T} \leq 0,
\end{equation}
or
\begin{equation}
	\int_I \frac{\dif Q}{T} \leq \left(-\int_B^A\frac{\dif Q}{T}\right)_R = \left(\int_A^B\frac{\dif Q}{T}\right)_R = S(B) - S(A).
\end{equation}
This implies that for an isolated system (i.e., $\dif Q=0$) entropy can only increase:
\[
	S(B) \geq S(A)
\]
where the equality holds for reversible processes only.
All processes in the real world are actually irreversible, at least to some extent.
Thus, the entropy of any real system always increases, even if that system is perfectly thermally insulated ($\dif Q=0$).

\subsection{Some Thermal Properties of Ideal Gasses}

We often approximate stellar plasmas as ideal gasses, fraught as that may be.
As we argued above, the internal energy is a function of temperature only.
So using the definition of the specific heat (Eq. \ref{e.specHeatP}) we have
\begin{equation}
	\varepsilon = \int C_V \dif T.
\end{equation}
While the proof will require some more development of kinetic theory, we will find that $C_V$ is a constant for an ideal gas, thus
\begin{equation}
	\varepsilon = C_V T.
\end{equation}



\section{The equations in Lagrangian form}

The fluid equations~(\ref{e.mass-1}), (\ref{e.momentum-2}), and (\ref{e.energy-2}) are in \textbf{Eulerian} form; that is, they describe everything in terms of spatial coordinates and time. This is not necessarily the most convenient form for  practical calculations. For example, the star can expand and contract, making the radius a function of time. Moreover, the velocity $\vu$ is \emph{not} the velocity of a given fluid element, which is why the equation of motion (eq.~[\ref{e.euler}]) is non-linear. It is often desirable to put the fluid equations into \textbf{Lagrangian} form, in which the coordinates are some label for a fluid element and time.

In one-dimension, the transformation to Lagrangian equations is easy.  At some reference time, we label the mass enclosed by a shell of radius $r$
\begin{equation}\label{e.mass}
	m(r,t) = \int_{0}^{r}\! \rho(r',t) 4\pi r'^{2} \,\dif r',
\end{equation}
as a Lagrangian coordinate $m$; we then transform coordinates from $(r,t)$ to $(m,t)$.
To do this, differentiate eq.~(\ref{e.mass}) w.r.t.\ $r$,
\[ \partial_{r}m = 4\pi r^{2}\rho, \]
and substitute for $\rho$ in the equation of  continuity (eq.~[\ref{e.mass-1}]).  The first term becomes
\[
	\partial_{t}\rho = \partial_{t}\left(\frac{1}{4\pi r^{2}} \partial_{r} m\right)
	= \frac{1}{4\pi r^{2}}\partial_{r}(\partial_{t}m),
\]
while the second term becomes
\[
	\frac{1}{4\pi r^{2}}\partial_{r}\left(u\partial_{r}m\right);
\]
the equation of continuity therefore becomes
\begin{equation}\label{e.mod-continuity}
	\frac{1}{4\pi r^{2}} \partial_{r}\left( \partial_{t} m + u\partial_{r} m\right) = 0.
\end{equation}
We can integrate this over $r$ to find that $\partial_{t} m + u\partial_{r} m = f(t)$, where $f(t)$ is some as-yet-unspecified function; to fix $f(t)$, we note that since $m(0,t) = 0,\;\forall t$, we must have $f(t) = 0$.  Now $\partial_{t} m + u\partial_{r} m = \Dif m/\Dif t = 0$, so along a streamline, $m$ is a constant.  We can therefore transform from coordinates $(r,t)$ to $(m,t)$ by setting
\begin{eqnarray}
	\label{e.lagrange-rule-1}
	\left.\frac{\partial}{\partial_{t}}\right|_{r} + u\left.\frac{\partial}{\partial r}\right|_{t}
	&=& \left.\frac{\partial}{\partial t}\right|_{m} \equiv \frac{\Dif}{\Dif t}\\
	\label{e.lagrange-rule-2}
	\left.\frac{\partial}{\partial r}\right|_{t} &=& 4\pi r^{2}\rho \left.\frac{\partial}{\partial m}\right|_{t}.
\end{eqnarray}
Here $\Dif/\Dif t \equiv (\partial/\partial t)_{m}$ is the Lagrangian time derivative.  In deriving this change, we used the equation of continuity, which becomes
\begin{equation}\label{e.lagrange-r}
\frac{\partial r}{\partial m} = \frac{1}{4\pi r^{2}\rho}.
\end{equation}
Our equation for momentum (eq.~[\ref{e.momentum-1}]) becomes
\begin{equation}\label{e.lagrange-momentum}
\frac{\partial P}{\partial m} = -\frac{Gm}{4\pi r^{4}} - \frac{1}{4\pi r^{2}}\frac{\Dif u}{\Dif t}.
\end{equation}
In hydrostatic balance the second term on the right-hand side is negligible.
The flux equation, (eq.~[\ref{e.conduction-simple}]) can be transformed to
\begin{equation}\label{e.lagrange-flux}
\frac{\partial T}{\partial m} = - \frac{1}{16\pi^{2} r^{4}\rho K}L_{r}
\end{equation}
Here $L_{r}$ is the luminous flux at a radius $r$.
%	\frac{\partial T}{\partial m} = -\frac{3}{64\pi^{2}r^{4}}\frac{\kappa}{ac T^{3}}L_{r}.

The energy equation (eq.~[\ref{e.energy-2}]) is more complicated. We can expand the time derivative as
\begin{eqnarray*}
	\partial_{t} \left( \frac{1}{2}\rho u^{2} + \rho \varepsilon \right)
	&=& \left(\frac{1}{2}u^{2} + \varepsilon\right)\partial_{t}\rho + \rho\partial_{t}\left[\frac{1}{2}(\vu\vdot\vu) + \varepsilon\right]\\
	&=& -\left(\frac{1}{2}u^{2} + \varepsilon\right)\divr\left(\rho\vu\right) + \rho \vu\partial_{t}\vu + \rho\partial_{t}\varepsilon,
\end{eqnarray*}
using equation~(\ref{e.mass-1}) to substitute for $\partial_{t}\rho$.  We then use equation~(\ref{e.momentum-1}) to replace $\partial_{t}\vu$, and recognizing that $\vu(\vu\vdot\grad)\vu = \vu\vdot\grad[(1/2)u^{2}]$, rewrite equation~(\ref{e.energy-2}) as
\begin{equation}\label{e.energy-3}
	\rho\left(\partial_{t} + \vu\vdot\grad\right) \varepsilon + P\divr\vu = -\divr\bvec{F} + \rho q.
\end{equation}
We've canceled all common factors here.  Finally, we once again use equation~(\ref{e.mass-1}) to set
\[
	P\divr\vu = -(P/\rho)(\partial_{t}\rho + \vu\vdot\grad \rho)
	= \rho P\left(\partial_{t} + \vu\vdot\grad\right)\left(\frac{1}{\rho}\right).
\]
Substituting this into the left-hand of equation~(\ref{e.energy-3})  and using the first law of thermodynamics (see eq.~[\ref{e.first-law-intensive}]), we obtain
\begin{equation}
\rho\left(\partial_{t} + \vu\vdot\grad\right) \varepsilon + P\divr\vu = \rho T \left(\partial_{t} + \vu\vdot\grad\right) s.
\end{equation}
For the right-hand side of equation~(\ref{e.energy-3}), we expand the divergence operator in spherical symmetry and use equation~(\ref{e.lagrange-rule-2}) to obtain
\[
	-\divr\bvec{F} = -\frac{1}{r^{2}}\frac{\partial(r^{2} F)}{\partial r} = -\rho\frac{\partial L_{r}}{\partial m}.
\]
Putting everything together, we finally have our heat equation in Lagrangian form,
\begin{equation}\label{e.lagrange-heat}
	\frac{\partial L_{r}}{\partial m} = q - T\frac{\Dif s}{\Dif t}.
\end{equation}
This has a simple interpretation: the change in luminosity across a mass shell is due to sources or sinks of energy and the change in the heat content of the shell.

\begin{exercisebox}[Lagrangian form of $T\Dif s/\Dif t$]
 \label{p.lagrange-heat} Show that equation (\ref{e.lagrange-heat}) can be written as
\begin{eqnarray}
\frac{\partial L_{r}}{\partial m} &=& q - \frac{c_{\rho} T}{\chi_{T}}\left\{\frac{\Dif\ln P}{\Dif t} - \left[\chi_{\rho} + \chi_{T}\left(\Gamma_{3} - 1\right)\right]\frac{\Dif\ln \rho}{\Dif t}\right\}\\
 &=& q - \frac{P}{\rho(\Gamma_{3}-1)}\frac{\Dif }{\Dif t}\ln\left(\frac{P}{\rho^{\Gamma_{1}}}\right),
\label{e.lagrange-heat-alt2}
\end{eqnarray}
where
\begin{eqnarray*}
 \chi_{T} &\equiv& \frac{T}{P}\tderiv{P}{T}{\rho},\\
 \chi_{\rho} &\equiv& \frac{\rho}{P}\tderiv{P}{\rho}{T},\\
 \Gamma_{1} &\equiv& (\partial \ln P/\partial\ln \rho)_{s}\,,\textrm{and}\\
 \Gamma_{3}-1 &\equiv& \tderiv{\ln T}{\ln \rho}{s}
\end{eqnarray*}
are defined in Appendix~\ref{s.thermo-derivatives}.
\end{exercisebox}


It is more useful, however, to work with temperature and pressure instead of entropy.  Write
\[
	T\frac{\Dif s}{\Dif t} = T\tderiv{s}{T}{P}\frac{\Dif T}{\Dif t} + T\tderiv{s}{P}{T}\frac{\Dif P}{\Dif t},
\]
and use the identity (see Appendix~\ref{s.thermo-derivatives})
\[
	\tderiv{s}{P}{T} = -\tderiv{s}{T}{P}\tderiv{T}{P}{s}
\]
to obtain
\begin{equation}\label{e.lagrange-heat-alt}
	\frac{\partial L_{r}}{\partial m}
	= q - c_{P}\left[ \frac{\Dif T}{\Dif t} - \tderiv{T}{P}{s} \frac{\Dif P}{\Dif t}\right].
\end{equation}
Equations~(\ref{e.lagrange-r}), (\ref{e.lagrange-momentum}), (\ref{e.lagrange-flux}), and (\ref{e.lagrange-heat-alt}), when supplemented by an equation of state, a prescription for the thermal conductivity, and the equations for nuclear heating and neutrino cooling, form the equations for stellar structure and evolution in spherical symmetry.
